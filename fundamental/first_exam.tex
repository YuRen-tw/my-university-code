\documentclass[12pt,a4paper]{article}

\usepackage{amsmath,amssymb,amsthm,mathtools}
\usepackage{rsfso}

\usepackage{geometry}
\geometry{top=10pt}
\pagenumbering{gobble}

%% Chinese
\usepackage{fontspec}
\newfontfamily\CJK{Noto Serif CJK KR}[BoldFont=Noto Serif CJK KR Bold]
% Google's Noto CJK Font
% https://www.google.com/get/noto/help/cjk/

\theoremstyle{definition}
\newtheorem{Ex}{}

\newcommand{\set}[1]{\{#1\}}
\renewcommand{\empty}{\varnothing}

\begin{document}

\title{\CJK%
{\normalsize 臺北市立大學數學系~105~學年第二學期}\\[4pt]
\textbf{基礎數學}\\
{\large 第一次期中考試題}{\small \S2.1-\S3.1}%
}
\author{}
\date{\vspace{-10ex}}
\maketitle

\renewcommand{\baselinestretch}{1.5}
\renewcommand{\labelenumi}{(\arabic{enumi})}
\begin{Ex}
Let $A$, $B$, $C$, and $D$ be sets. Prove that
\begin{enumerate}
  \item $\mathcal{P}(A \cap B) = \mathcal{P}(A) \cap \mathcal{P}(B)$. \quad (8~points)
  \item $(A \times B) \cap (C \times D) = (A \cap C) \times (B \cap D)$. \quad (8~points)
\end{enumerate}
\end{Ex}

\begin{Ex}
Let $\mathcal{A} = \set{A_\alpha : \alpha\in\Delta}$ be a family of sets, $\Delta\ne\empty$, and $B$ be a set. Prove that
\begin{enumerate}
  \item $\bigl(\bigcap_{A\in\mathcal{A}}A\bigr)^c = \bigcup_{A\in\mathcal{A}}A^c$. \quad (9~points)
  \item $\bigl(\bigcup_{\alpha\in\Delta}A_\alpha\bigr) - B = \bigcup_{\alpha\in\Delta}(A_\alpha-B)$. \quad (9~points)
\end{enumerate}
\end{Ex}

\begin{Ex}
Use the PMI to prove that for every natural number $n$, $\displaystyle\frac{7}{15}n+\frac{1}{3}n^3+\frac{1}{5}n^5$ is an integer. \quad (9~points)
\end{Ex}

\begin{Ex}
Prove that the Well-Ordering Pinciple (WOP) implies the Principle of Complete Induction (PCI). \quad (9~points)
\end{Ex}

\begin{Ex}
For each natural number $n$, the $n$-th Fibonacci number $f_n$ is defined by
\[
  f_1 = 1,\ f_2 = 1,\ \text{and } f_{n+2} = f_{n+1} + f_n \text{ for } n \ge 1.
\]
Use the PCI to prove that $f_{n+6} = 4f_{n+3} + f_n$ for all natural number $n$. \quad (9~points)
\end{Ex}

\begin{Ex}
Suppose that $A$, $B$, $C$, and $D$ are sets. Let $R$ be a relation from $A$ to $B$, $S$ be a relation from $B$ to $C$, and $T$ be a relation from $C$ to $D$. Prove that $T\circ(S\circ R) = (T\circ S)\circ R$ and $(S\circ R)^{-1} = R^{-1} \circ S^{-1}$. \quad (12~points)
\end{Ex}

\begin{Ex}
Let $\mathcal{A}$ and $\mathcal{B}$ be two pairwise disjoint families of sets. Suppose that $\bigcup_{A\in\mathcal{A}}A$ and $\bigcup_{B\in\mathcal{B}}B$ are disjoint. Prove that $\mathcal{A}\cup\mathcal{B}$ is pairwise disjoint. \quad (12~points)
\end{Ex}

\begin{Ex}
Suppose that $a\in\mathbb{N}$ and $b\in\mathbb{Z}$. Prove that there exists unique integers $q$ and $r$ such that $b=aq+r$ and $0\le r < a$. \quad (15~points)
\end{Ex}
\end{document}

